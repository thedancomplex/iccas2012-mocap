\section{Methodology}

Mapping human motion to humanoid motion incorporates the following stages: 1. Human motion capture. 2. Kinematics mapping.   3. Dynamic mapping. 
The motion capture system used in the experiment consists of eighteen V100:R2 cameras. It is able to record the full body motion using thirty-eight reflective markers placed on the actor's body. The system creates a 3D cloud of marker points at the rate of one hundred frames per second in sub-millimeter precision. These 3D points are labeled and mapped to a skeleton of twenty two bones with specifications of the actor's body. So the skeleton is a human body representation in the motion capture system. In this way, the position and the rotation of the bones are obtained in each frame. The rotation of each bone is given in quaternions with respect to the end tip of the previous bone.
Kinematics mapping has to resolve three key issues: 
1)	The difference in the total degree of freedom (DOF). 
2)	The difference in the kinematics descriptions. 
3)	Kinematics constraints for the robot.
The difference in number of bones (linkages) as well as the DOF for each joint caused the difference in the total DOF. 
In reality, all anatomic joints for a human have six DOF []. Humans also have soft tissues and vertebra that create more DOF []. The skeleton generated does not have soft tissues, it replaced the vertebra with three bones and Jaemi HUBO has only one DOF(yaw) in between the neck and the hip. The key to a successful kinematics mapping will then be to deliver the main information as close as possible and estimate the rest. Hence for the underhand throwing motion, the yaw angle of Jaemi HUBO's waist is the combination of the two joints' yaw angles of the human skeleton. 
The difference in kinematics descriptions lies in the joint configuration. (The gross movement of the limb segments interconnected by joints where for the Jaemi HUBO the relative joint rotation is described by adopting the Eulerian angle system, for the human skeleton, Quaternions are used.) For Jaemi HUBO, the finite rotation of joint in the two bone segments is sequence dependent while for human skeleton, the associated three-dimensional finite rotation is sequence independent.
Unlike the generated skeleton in the motion capture system where each joint has multiple DOF, Jaemi HUBO is constructed with single DOF joints. This makes the joint orientations follow a specific sequence during conversion. For example, from the shoulder to the upper arm, the sequence for shoulder joints is pitch, roll and yaw. (If we move roll prior to pitch, it won't affect the pitch axis but will change the yaw axis.)


(pix here)

Kinematics constraints for the robot include joint angle range and limb contact. 
The kinematics mapping starts from the conversion from quaternions used in motion capture generated skeleton to Euler angles used in Jaemi HUBO. We mainly solve two problems during conversion.
1.	Unlike quaternion representation, Euler angles are determined by 24 conventions. As discussed in key issues 2), the joint orientation matters for the robot. Therefore to generate the mapping for a given set of Euler angles, we map rotating axis ri, rj, rk to reference axis i, j, k in the appropriate order. 
2.	Singularities are generated in conversion from quaternions to Eulerian angle system. When the pitch angle approaches +90 degrees, singularities may occur. We use two methods to avoid singularities. First, use an approximated function for the singularity zone. Second, choose a favorable sequence for joints that have less than three orientations for the robot.
Convention zxy: 
double test = qw*qx + qy*qz;
if (test > 0.499) // singularity at north pole
		yaw = (float) 2.0f * atan2(qy,qw);
		pitch = 3.14159265f/2.0f;
		roll = 0;
		
if (test < -0.499) // singularity at south pole
		yaw = (float) -2.0f * atan2(qy,qw);
		pitch = - 3.14159265f/2.0f;
		roll = 0;
        
	
    double sqx = qx*qx;
    double sqy = qy*qy;
    double sqz = qz*qz;
yaw = (float) atan2((double)2.0*qw*qz-2.0*qx*qy , (double)1 - 2.0*sqz - 2.0*sqx);
pitch = (float)asin(2.0*test);
roll = (float) atan2((double)2.0*qw*qy-2.0*qx*qz , (double)1.0 - 2.0*sqy - 2.0*sqx);
